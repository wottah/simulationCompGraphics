\chapter{Forces}

\section{Gravity force}
The following formula $\boldsymbol{f}_{grav} = m*\boldsymbol{G}$ is implemented with the following code.
\begin{lstlisting}
particles.ForEach(x => x.ForceAccumulator += gravity * x.Massa;
\end{lstlisting}

\section{Point gravity force}
A point gravity can be described as $f_{grav} = \boldsymbol{D}*(f(l_{distance})*m_{particle}*m_{center} \max 0)$, where D is the direction to the center, f is a polynomial function and $m_{particle}$ and $m_{center}$ are the masses. This is implemented in the following code (g is are scalars of the polynomial function).
\begin{lstlisting}
float distance = (center.Position-p.Position).GetLength();
if(distance == 0)
{
	return new Vector(0, 0);
}
float delta = 0;
for (int i = 0; i < g.Dim; i++)
{
	delta += Math.Pow(distance, i) * g[i]);
}
Vector direction = (center.Position - p.Position).Normilize();
return direction * Math.Max(0, delta*p.Massa*center.Massa);
\end{lstlisting}

\section{Spring force}
The following formula is used for implementing a spring force $f_{p1} = [{k_s(|\boldsymbol{l}| - r) + k_d \frac{\boldsymbol{\dot{l}}\cdot\boldsymbol{l}}{|\boldsymbol{l}|}}]\frac{1}{|\boldsymbol{l}|}\boldsymbol{l}$ and $f_{p1} -= f_{p2}$.
\begin{lstlisting}
Vector l = p1.Position - p2.Position;
Vector lDot = p1.Velocity - p2.Velocity;
float lAbs = l.GetLength();
Vector springforce = (l / lAbs)*((lAbs - _r) * _ks + (Dot(lDot, l) / lAbs) * _kd);
p1.ForceAccumulator -= springforce;
p2.ForceAccumulator += springforce;
\end{lstlisting}

\section{Mouse spring force}
The mouse spring is the same as a spring force, but only to a constant point(the mouse position) and the force is only 1 sided.

\section{Viscous drag Force}
The following formula $\boldsymbol{f}_{drag} = -k_{drag}*\boldsymbol{v}$ is implemented with the following code.
\begin{lstlisting}
particles.ForEach(x => x.ForceAccumulator -= _drag * x.Velocity);
\end{lstlisting}