\chapter{Integration} 
Every integration scheme uses a blackbox to calculate the forces and constraints, and subsequently the velocity's.

\section{Euler integration}
The euler integration is implemented as follows.
\begin{lstlisting}
blackbox.Execute(particles, dt);
particles.ForEach(x => x.Position += x.Velocity * dt);
\end{lstlisting}
Which implements the following formula $q = \dot{q}*\Delta t$.

\section{Mid Point integration}
The midpoint integration scheme is implemented as follows

\begin{lstlisting}
//Copy system to keep the original position
List<Particle> particlesCopy = particles.Copy();

//first calculate midpoint
blackbox.Execute(particlesCopy, dt);
particlesCopy.ForEach(x => x.Position += x.Velocity * (dt/2));

//calculate forces on midpoint
blackbox.Execute(particlesCopy, dt);

for (int i = 0; i < particles.Count; i++)
{
	particles[i].Velocity = particlesCopy[i].Velocity;
	particles[i].Position += particlesCopy[i].Velocity*dt;
}
\end{lstlisting}

\section{Runge Kutta integration}
The following code implements the runga kutta integration scheme of order 4. These scheme is described in formule \ref{formula:RK41} through \ref{formula:RK45}.

\begin{lstlisting}
//copy system for to keep original positions
List<Particle> k1 = particles.ConvertAll(x => x.Clone());
List<Particle> k2 = particles.ConvertAll(x => x.Clone());
List<Particle> k3 = particles.ConvertAll(x => x.Clone());
List<Particle> k4 = particles.ConvertAll(x => x.Clone());

blackbox.Execute(k1, dt);

for (int i = 0; i < particles.Count; i++)
{
	k2[i].Velocity = k1[i].Velocity;
	k2[i].Position += k1[i].Velocity * dt;
}

blackbox.Execute(k2, dt/2);

for (int i = 0; i < particles.Count; i++)
{
	k3[i].Velocity = k2[i].Velocity;
	k3[i].Position += k2[i].Velocity * dt;
}

blackbox.Execute(k3, dt);

for (int i = 0; i < particles.Count; i++)
{
	k4[i].Velocity = k3[i].Velocity;
	k4[i].Position += k3[i].Velocity * dt;
}

blackbox.Execute(k4, dt);

for (int i = 0; i < particles.Count; i++)
{
	particles[i].Velocity = k1[i].Velocity/6 + k2[i].Velocity/3 + k3[i].Velocity/3 + k4[i].Velocity/6;
	particles[i].Position += particles[i].Velocity*dt;
}
\end{lstlisting}

\begin{eqnarray}
\label{formula:RK41}
k_1 &=& hf(x_0, t_0) \\
k_2 &=& hf(x_0 + k_1/2, t_0+h/2) \\
k_3 &=& hf(x_0 + k_2/2, t_0+h/2) \\
k_4 &=& hf(x_0 + k_3/2, t_0+h) \\
\label{formula:RK45}
x(t_0 + h) &=& x_0 + \frac{1}{6}k_1 + \frac{1}{3}k_2 + \frac{1}{3}k_3 + \frac{1}{6}k_4
\end{eqnarray}

\section{Verlet integration}
Lastly we implemented the Verlet integration scheme. This is described in the following formula.

\begin{eqnarray*}
\boldsymbol{x_1} &=& \boldsymbol{x_0} + \boldsymbol{\dot{x_0}}\Delta t + \frac{1}{2} \boldsymbol{\ddot{x_0}}\Delta t^2 \\
\boldsymbol{x_{n+1}} &=& 2\boldsymbol{x_n} - \boldsymbol{x_{n-1}}\Delta t + \boldsymbol{\ddot{x_0}}\Delta t^2 \
\end{eqnarray*}
This is implemented in the following code. Note that \_previousState starts with a null value.
\begin{lstlisting}
List<Particle> tmp = particles.ConvertAll(x => x.Clone());
if (_previousState == null)//first time
{
	blackbox.Execute(particles, dt);
	particles.ForEach(x => x.Position = x.Position + x.Velocity*dt + 0.5f*x.ForceComplete*dt*dt);
}
else //after the first time
{
	blackbox.Execute(particles, dt);
	for (int i = 0; i < particles.Count; i++)
	{
		particles[i].Position = 2*particles[i].Position - _previousState[i].Position +
		                        particles[i].ForceComplete*dt*dt;
	}
}
_previousState = tmp;
\end{lstlisting}
